\documentclass[12pt,a4paper]{article}
	\usepackage[T1]{fontenc}
	\usepackage{enumerate} % Needed for markdown enumerations to work
	\usepackage{geometry} % Used to adjust the document margins
	\usepackage{amsmath} % Equations
	\usepackage{amssymb} % Equations
	\usepackage{textcomp} % defines textquotesingle
    \usepackage[utf8x]{inputenc} % Allow utf-8 characters in the tex document
	\usepackage[T2A]{fontenc}
	\usepackage[english, russian]{babel}
%	\usepackage{mathtools}


\title{Прямое численное моделирование распространения пламени}

\begin{document}

\maketitle

\section{Постановка задачи прямого моделирования в программном пакете AST FIRE}
 
 Задача определена в условиях постоянства давления и постоянства плотности. Турбулентность в скорости определена мгнновенными реализациями случайных распределений скоростей при средней скорости $\bar{u}$ и девиацией $u'$. Здесь получается, что уравние неразрысности и уравнения сохранения момента движения не решаются.
 
 Составим систему уравнений в общем виде:
\begin{itemize}
     \item уравнение неразрывности
        \begin{equation}\label{eq_01}
        \frac{\partial \rho }{\partial t} + 
        \nabla\cdot \rho \boldsymbol{u} = 0
        \end{equation}
    
    \item уравнение сохранения компонент смеси
         \begin{equation}\label{eq_02}
        \frac{\partial \rho Y_i}{\partial t} + 
        \nabla\cdot \left(\rho \boldsymbol{u} Y_i \right) = 
        \nabla \cdot \left( D \nabla Y_i \right)
        \end{equation}

    \item уравнение сохранения энергии
         \begin{equation}\label{eq_03}
        \frac{\partial \rho H}{\partial t} + 
        \nabla\cdot \left(\rho \boldsymbol{u} H \right) = 
        \nabla \cdot \left( \lambda \nabla T \right)
        \end{equation}
 \end{itemize}
 
 
 Проведем преобразование уравнение неразрывности компонент в согласии предложенной постановкой. Раскроем множители в левой части уравнения 

 \begin{equation}\label{eq_04}
\rho \frac{\partial Y_i}{\partial t} + 
\rho \boldsymbol{u}\cdot \nabla Y_i + 
 Y_i \cdot \left( \frac{\partial \rho}{\partial t} + \nabla\cdot \rho \boldsymbol{u} \right) = 
 	\nabla \cdot \left( D \nabla Y_i \right)
\end{equation}

При предположении, что $\rho = const$ уравнение примет вид

\begin{equation}\label{eq_05}
\rho_0 \frac{\partial Y_i}{\partial t} + 
\rho_0 \boldsymbol{u}\cdot \nabla Y_i  + 
Y_i \rho_0 \nabla\cdot \boldsymbol{u} = 
 	\nabla \cdot \left( D \nabla Y_i \right)
\end{equation}


Третий член в уравнении \eqref{eq_05} $Y_i \rho_0 \nabla\cdot  \boldsymbol{u}$ должен быть нулевым, что требует выполнения уравнения неразрывности. По алгоритму программы  уравнение неразрывности выполняется при реализации мгновенных пульсаций скоростей.

При преобразовании уравнения сохранения энергии \eqref{eq_03} получим  похожую форму уравнения

\begin{equation}\label{eq_06}
\rho_0 \frac{\partial H}{\partial t} + 
\rho_0 \boldsymbol{u}\cdot \nabla H  + 
H  \rho_0 \nabla\cdot \boldsymbol{u} = 
\nabla \cdot \left( \lambda \nabla T \right)
\end{equation}

В уравнении \eqref{eq_06} получаем член $H  \rho_0 \nabla\cdot \boldsymbol{u}$, который будет нулевым только в случае выполнения уравнения реразрывности. 

%\section{Проверка реализации}
%
%Проведем проверку функций корреляции при реализации пульсаций. используем формулу
%
%\begin{equation}\label{eq_07}
%R(x,x+\Delta) = <\boldsymbol{u}(x)\cdot \boldsymbol{u}(x+\Delta)>
%\end{equation}
\section{Постановка задачи}
Поле скоростей разделяется на два компонента
\begin{equation} \label{eq_u_main}
\boldsymbol{u} = \langle \boldsymbol{u} \rangle + \boldsymbol{u'} \\ ,
\end{equation}

где $ \langle u \rangle $ средняя скорость  и $u'$ - пульсационная скорость . Известно, что среднее значение пульсационной составляющей сокрости $\langle u`\rangle = 0 $.
Для воспроизведения изотропной турбулентности сгенерируем поле пульсационных скоростей по формуле

\begin{equation} \label{eq_u_initial}
\boldsymbol{u}(\boldsymbol{x},t) = 
\sum_{n=1}^N \sum_{m=1}^M \boldsymbol{p}^{n,m} 
\cos{\left( \boldsymbol{k}^{n,m} \cdot \boldsymbol{x} + w^{n,m} t \right)} + 
\boldsymbol{q}^{n,m} \sin{\left( \boldsymbol{k}^{n,m} 
	\cdot \boldsymbol{x} + w^{n,m} t \right)}
\end{equation}

где $\boldsymbol{p}^{n,m}$ и $\boldsymbol{q}^{n,m}$ --- векторный величины амплитудой соответствующей спектру кинетической турбулентной энергии и случайным направлением, $\boldsymbol{k}^{n,m}$ --- вектор волнового числа в случайном направлении 

Для выполнения уравнения неразрывности несжимаемой среды требуется выполнение уравнения $\nabla \cdot \boldsymbol{u} = 0$ и ввести ограничения на свободные переменные. Решение уравнения неразрывности в непрерывном виде будет иметь вид.

\begin{equation} \label{eq_u_initial}
\boldsymbol{u}(\boldsymbol{x},t) = 
\sum_{n=1}^N \sum_{m=1}^M \boldsymbol{p}^{n,m} 
\cos{\left( \boldsymbol{k}^{n,m} \cdot \boldsymbol{x} + w^{n,m} t \right)} + 
\boldsymbol{q}^{n,m} \sin{\left( \boldsymbol{k}^{n,m} 
	\cdot \boldsymbol{x} + w^{n,m} t \right)}
\end{equation}

\section*{Вывод}

\section{Генерация поля скоростей}

Для генерации поля скоростей изотропной турбулентности используем подход 


Обязательно требуется замыкание скоростей на уравнение неразрывности при реализации случайного распределения скоростей.

\end{document}
