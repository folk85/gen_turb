\documentclass[12pt,a4paper]{article}
	\usepackage[T1]{fontenc}
	\usepackage{enumerate} % Needed for markdown enumerations to work
	\usepackage{geometry} % Used to adjust the document margins
	\usepackage{amsmath} % Equations
	\usepackage{amssymb} % Equations
	\usepackage{textcomp} % defines textquotesingle
    \usepackage[utf8x]{inputenc} % Allow utf-8 characters in the tex document
	\usepackage[T2A]{fontenc}
	\usepackage[english, russian]{babel}
%	\usepackage{mathtools}


\title{Прямое численное моделирование распространения пламени}

\begin{document}

\maketitle

\section{Постановка задачи прямого моделирования в программном пакете AST FIRE}
 
 Задача определена в условиях постоянства давления и постоянства плотности. Турбулентность в скорости определена мгнновенными реализациями случайных распределений скоростей при средней скорости $\bar{u}$ и девиацией $u'$. Здесь получается, что уравние неразрысности и уравнения сохранения момента движения не решаются.
 
 Составим систему уравнений в общем виде:
\begin{itemize}
     \item уравнение неразрывности
        \begin{equation}\label{eq_01}
        \frac{\partial \rho }{\partial t} + 
        \nabla\cdot \rho \boldsymbol{u} = 0
        \end{equation}
    
    \item уравнение сохранения компонент смеси
         \begin{equation}\label{eq_02}
        \frac{\partial \rho Y_i}{\partial t} + 
        \nabla\cdot \left(\rho \boldsymbol{u} Y_i \right) = 
        \nabla \cdot \left( D \nabla Y_i \right)
        \end{equation}

    \item уравнение сохранения энергии
         \begin{equation}\label{eq_03}
        \frac{\partial \rho H}{\partial t} + 
        \nabla\cdot \left(\rho \boldsymbol{u} H \right) = 
        \nabla \cdot \left( \lambda \nabla T \right)
        \end{equation}
 \end{itemize}
 
 
 Проведем преобразование уравнение неразрывности компонент в согласии предложенной постановкой. Раскроем множители в левой части уравнения 

 \begin{equation}\label{eq_04}
\rho \frac{\partial Y_i}{\partial t} + 
\rho \boldsymbol{u}\cdot \nabla Y_i + 
 Y_i \cdot \left( \frac{\partial \rho}{\partial t} + \nabla\cdot \rho \boldsymbol{u} \right) = 
 	\nabla \cdot \left( D \nabla Y_i \right)
\end{equation}

При предположении, что $\rho = const$ уравнение примет вид

\begin{equation}\label{eq_05}
\rho_0 \frac{\partial Y_i}{\partial t} + 
\rho_0 \boldsymbol{u}\cdot \nabla Y_i  + 
Y_i \rho_0 \nabla\cdot \boldsymbol{u} = 
 	\nabla \cdot \left( D \nabla Y_i \right)
\end{equation}


Третий член в уравнении \eqref{eq_05} $Y_i \rho_0 \nabla\cdot  \boldsymbol{u}$ должен быть нулевым, что требует выполнения уравнения неразрывности. По алгоритму программы  уравнение неразрывности выполняется при реализации мгновенных пульсаций скоростей.

При преобразовании уравнения сохранения энергии \eqref{eq_03} получим  похожую форму уравнения

\begin{equation}\label{eq_06}
\rho_0 \frac{\partial H}{\partial t} + 
\rho_0 \boldsymbol{u}\cdot \nabla H  + 
H  \rho_0 \nabla\cdot \boldsymbol{u} = 
\nabla \cdot \left( \lambda \nabla T \right)
\end{equation}

В уравнении \eqref{eq_06} получаем член $H  \rho_0 \nabla\cdot \boldsymbol{u}$, который будет нулевым только в случае выполнения уравнения реразрывности. 

%\section{Проверка реализации}
%
%Проведем проверку функций корреляции при реализации пульсаций. используем формулу
%
%\begin{equation}\label{eq_07}
%R(x,x+\Delta) = <\boldsymbol{u}(x)\cdot \boldsymbol{u}(x+\Delta)>
%\end{equation}
\section{Постановка задачи}

В работе \cite{kraichnan_diffusion_1970} предложено использовать уравнение поля скоростей \ref{eq_u_kraichman} , которая стаистически воспроизводит спектр изотропной турбулентности. 
\begin{equation} \label{eq_u_kraichman}
\boldsymbol{u}(\boldsymbol{x}) = 2 \cdot \sum_{n=1}^{N} \boldsymbol{p}^{n} \cos \left(\boldsymbol{k}^n\cdot \boldsymbol{x}+\psi^n\right)
\end{equation}

В уравнении \ref{eq_u_kraichman} определены $N$ частотных мод. Уравнение неразрывности выполняется условием перпендикулярности векторов $\boldsymbol{k}^n \perp \boldsymbol{p}^n $ . Вектора $\boldsymbol{k}^n$ и $\boldsymbol{p}^n$ представляются заданной амплитуды и единичным вектором направления. В работе предложена реализация  модели с изменением по времени.

В заметке \cite{saad_comment_2016}  к оригинальной работе \cite{kraichnan_diffusion_1970} проведен анализ подхода и предложен другой подход к представлению уравнения. Уравнение (\ref{eq_u_initial}) дополнено вторым слагаемым и обеспечивает выполнение уравнения неразрывности для дискретного вида. 

\begin{equation} \label{eq_u_initial}
\boldsymbol{u}(\boldsymbol{x},t) = 
\sum_{n=1}^N \sum_{m=1}^M \boldsymbol{p}^{n,m} 
\cos{\left( \boldsymbol{k}^{n,m} \cdot \boldsymbol{x} + w^{n,m} t \right)} + 
\boldsymbol{q}^{n,m} \sin{\left( \boldsymbol{k}^{n,m} 
	\cdot \boldsymbol{x} + w^{n,m} t \right)}
\end{equation}

где $\boldsymbol{p}^{n,m}$ и $\boldsymbol{q}^{n,m}$ --- векторный величины амплитудой соответствующей спектру кинетической турбулентной энергии и случайно определнному единичному вектору направления, $\boldsymbol{k}^{n,m}$ --- вектор волнового числа в случайном направлении. 


Для выполнения уравнения неразрывности несжимаемой среды требуется выполнение уравнения $\nabla \cdot \boldsymbol{u} = 0$ и ввести ограничения на свободные переменные. Решение уравнения неразрывности в непрерывном виде будет иметь вид.

\begin{eqnarray} \label{eq_u_mass_con}
\nabla\boldsymbol{u}(\boldsymbol{x},t) = 
\sum_{n=1}^N \sum_{m=1}^M \left[  \boldsymbol{p}^{n,m}
 \cdot \left(- \boldsymbol{k}^{n,m}\right)  
\sin{\left( \boldsymbol{k}^{n,m} \cdot \boldsymbol{x} + w^{n,m} t \right)} 
	+ \boldsymbol{q}^{n,m} 
\cdot \left( \boldsymbol{k}^{n,m}\right)
\cos{\left( \boldsymbol{k}^{n,m} 
	\cdot \boldsymbol{x} + w^{n,m} t \right)}  \right]
\end{eqnarray}

Для выполнения условий уравнения неразрывности требуется выполнение ортогональности векторов $\boldsymbol{p}^{n,m}$ и $\boldsymbol{k}^{n,m}$ для первого слагаемого и $\boldsymbol{q}^{n,m}$ и $\boldsymbol{k}^{n,m}$ для второго слагаемого. Получаем дополнительные выражения 

\begin{equation}
\boldsymbol{p}^n \cdot \boldsymbol{k}^n = 0\ ;\ \boldsymbol{q}^n \cdot \boldsymbol{k}^n = 0
\end{equation}


% Получаем требование ортогональности для векторов $\boldsymbol{k}^{n,m}\perp \boldsymbol{p}^{n,m}\perp \boldsymbol{q}^{n,m}$. 


Предложенное выражение \ref{eq_u_initial} для генерации поля скоростей можно применять на любой сетке. При численной реализации уравнение неразрывности будет выполняться с небольшой ошибкой. Также предложенное уравнение скоростей не выполняет граничные условия (специальные или периодические). Однако доля таких ячеек будет сравнительно малой. При генерации трехмерного поля $N\times N\times N$ и периодической границе на трех сторонах получим $3 N^2$ особых ячеек. Получается доля $3 / N$ от всех ячеек. В нашем случае с 2 периодическими границами получаем долю ячеек $2 / N$ \footnote{Численная оценка ошибки от применения периодических границ будет приведена позже.}.




В случае использования структурированной ортогональной сетки выполнение уравнения неразрывности в дискретном виде будет иметь другой вид. Представим выполнение УН в дискретном  виде по оси Х.
\begin{equation} \label{eq_continum_discrete}
	\nabla_x \boldsymbol{u}_{i,j,k} = \frac{1}{\Delta_x}\left( u_{i+1/2,j,k}-u_{i-1/2,j,k}\right)
\end{equation}

Поставим уравнение \ref{eq_u_initial}  в уравнение неразрывности в виде \ref{eq_continum_discrete}. Далее путем тригонометрических преобразований получим скорректированное условие для векторов  $\boldsymbol{\tilde{k}}^{n,m}\perp \boldsymbol{p}^{n,m}\perp \boldsymbol{q}^{n,m}$. Здесь вектор  $\boldsymbol{{k}}$ преобразуется на дискретном пространстве в 
\begin{equation} \label{eq_tilde_k}
\boldsymbol{\tilde{k}} = \left\{ 
\frac{2}{\Delta_x} \sin{\left( k_x \frac{\Delta_x}{2}  \right)},
\frac{2}{\Delta_y} \sin{\left( k_y \frac{\Delta_y}{2}  \right)},
\frac{2}{\Delta_z} \sin{\left( k_z \frac{\Delta_z}{2}  \right)} \right\}
\end{equation}

Видно, что в дискретном представлении  $\tilde{\boldsymbol{k}}\neq \boldsymbol{k}$, если не выполняется условие $\left(\Delta_x , \Delta_y, \Delta_z\right) \rightarrow 0$, и не выполняется уравнение неразрывности в дискретном виде. В работе \cite{bailly_stochastic_1999} впервые сделано замечание об этом.  

В работе \cite{saad_comment_2016} указано, что применение численных схем высокого порядка может снизить уровень численной ошибки и предложено использовать вариант определения вектора $\tilde{\boldsymbol{k}}$ в виде\footnote{Верность данного утверждения будет проверена ниже.}

\begin{equation} \label{eq_tilde_k_horder}
\boldsymbol{\tilde{k}} = \left\{ 
\frac{1}{\Delta_x} \sin{\left( k_x {\Delta_x}  \right)},
\frac{1}{\Delta_y} \sin{\left( k_y {\Delta_y}  \right)},
\frac{1}{\Delta_z} \sin{\left( k_z {\Delta_z}  \right)} \right\}.
\end{equation}

Для генерации поля по времени и пространству используется переменная $w^n = 2\pi f_n$, где $f_n = u' \cdot \boldsymbol{k}^n$. 

\section{Реализация генерации поля}
 Генерация поля происходит по следующей схеме:
\begin{enumerate}
\item Требуется определить размеры расчетной области и количество ячеек по каждому направлению в структурированной сетке.
\item Определим минимальную и максимальную частоту в соответствии с размером ячейки и масштабом расчетной области.
\item Определим количество мод и распределения частот.
\item Определяем случайный единичный вектор по модели:
	\begin{eqnarray}
	\boldsymbol{\xi} = \left[\xi_x, \xi_y, \xi_z\right] \\
	\xi_x = \cos \theta \cdot \cos \psi \nonumber\\ 
	\xi_y = \cos \theta \cdot \sin \psi  \nonumber\\ 
	\xi_z = \sin \theta \nonumber
	\end{eqnarray} 
	Определяем единичный вектор в полярной стсиеме координат. Здесь требуется задать два случайных числа $\theta \in N[0,2\pi)$ и $\psi \in N[-\pi,\pi)$.
\item Вычисляем амплитуды векторов 
\item 
\item 
\item 
\item 
\item 
\end{enumerate}
Представленное условие снижает отклонения при генерации поля скоростей. 



Данное условие довольно легко задать. Определяем случайное направление вектора $\boldsymbol{k}^{n,m}$, далее задаем первый единичный вектор случайным распределением. Для вектора $\boldsymbol{\tilde{p}}^{n,m}$ требуем выполнение ортогональности 
\begin{equation}
 	\boldsymbol{\tilde{p}}^{n,m} = \frac{\boldsymbol{\xi} \times \boldsymbol{k}^{n,m} }{\left\|\boldsymbol{\xi} \times \boldsymbol{k}^{n,m} \right\|}
\end{equation}

\begin{equation}
 	\boldsymbol{\tilde{q}}^{n,m} = \frac{\boldsymbol{\tilde{p}}^{n,m} \times \boldsymbol{k}^{n,m} }{\left\|\boldsymbol{\tilde{p}}^{n,m} \times \boldsymbol{k}^{n,m} \right\|}
\end{equation}

\section*{Вывод}

\section{Генерация поля скоростей}

Для генерации поля скоростей изотропной турбулентности используем подход 


Обязательно требуется замыкание скоростей на уравнение неразрывности при реализации случайного распределения скоростей.

\medskip

%Sets the bibliography style to UNSRT and imports the 
%bibliography file "samples.bib".
\bibliographystyle{plain}
\bibliography{refers}

\end{document}
